\documentclass[fleqn]{article}
\usepackage{amsmath}

\begin{document}
\thispagestyle{empty}

\subsection*{Descriptions}
\noindent
The slope of one line, is the reciprocal of the slope of its inverse line.
\newline
Let's say we have function $f(x)$ and its inverse as below:

\begin{align*}
f(x) &= \frac{1}{2}x + 1 \\
f^{-1}(x) &= 2x - 2
\end{align*}
\newline

\noindent
Then we have:

\begin{align*}
\frac{d}{dx}f(x) &= \frac{d}{dx}\left(\frac{1}{2}x + 1\right) = \frac{1}{2} \\
\frac{d}{dx}f^{-1}(x) &= \frac{d}{dx}(2x - 2) = 2
\end{align*}
\newline

\noindent
So, the derivatives are reciprocals of one another.
\newline

\subsection*{Exercises}
\noindent
Find $f^{-1}(x)$ in below exercises, then graph $f$ and $f^{-1}$ together:
\newline

\begin{equation*}
    f(x) = 2x + 3, \qquad a = -1
\end{equation*}

\begin{equation*}
    f(x) = 1/5(x) + 7, \qquad a = -1
\end{equation*}

\begin{equation*}
    f(x) = 5 - 4x, \qquad a = 1/2
\end{equation*}

\begin{equation*}
    f(x) = 2x^2 x \geq 0, \qquad a = 5
\end{equation*}
\end{document}
