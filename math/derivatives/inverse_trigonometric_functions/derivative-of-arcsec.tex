\documentclass[fleqn]{article}
\pagenumbering{gobble}
\usepackage{amsmath}

\begin{document}

\subsection*{Defintions}
\noindent
To find the derivative of $f{^{-1} (u)} = \sec{^{-1} x}$,
instead of using the theorem used for $\tan{^{-1} x}$ and ${\sin{^{-1} x}}$,
we find the derivative of $y = \sec{^{-1} x}, \mid{x}\mid > 1$,
using implicit differentiation and the Chain Rule as follows:
\newline

\begin{align*}
    y &= \sec{^{-1} x} \\
    \sec{y} &= x \qquad\qquad\qquad \text{Inverse function relationship} \\
    \frac{d}{dx}\left(\sec{y}\right) &= \frac{d}{dx}x \qquad\qquad\;\; \text{Differentiate both sides} \\
    \sec{y}\tan{y}\frac{dy}{dx} &= 1 \qquad\qquad\qquad \text{Chain rule} \\
    \frac{dy}{dx} &= \frac{1}{\sec{y}\tan{y}} \\
    \frac{dy}{dx} &= \pm \frac{1}{x \sqrt{x^2 - 1}} \quad \sec{y}
    = x \quad\text{and}\quad \tan{y} = \pm \sqrt{\sec{^2 y} - 1} = \pm \sqrt{x^2 - 1} \\
\end{align*}

\noindent
Finally we use absolue value sign to eliminate the "$\pm$" ambiguity:
\newline
\begin{equation*}
    \frac{d}{dx}\sec{^{-1} x} = \frac{1}{\mid{x}\mid \sqrt{x^2 - 1}}
\end{equation*}

\noindent
So, as a general rule for $f{^{-1} x} = \sec{^{-1} x}$, we have:
\newline
\begin{equation*}
    \frac{d}{dx}\left(\sec{^{-1} x}\right) = \frac{1}{\mid{u}\mid \sqrt{u^2 - 1}}\frac{du}{dx},\qquad\mid{u}\mid > 1
\end{equation*}

\subsection*{Exercises}
\noindent
Find the derivative of following excercises:
\newline

\begin{align*}
1.\;&\sec{^{-1} \left(2s+1\right)} \\
2.\;&\sec{^{-1} 5s} \\
3.\;&\sec{^{-1} \frac{1}{t}},\qquad{0<t<1}
\end{align*}

\end{document}
